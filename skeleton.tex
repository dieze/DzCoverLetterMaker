\documentclass[a4paper,12pt]{lettre}

\usepackage[francais]{babel}
\usepackage[utf8]{inputenc}
\usepackage[T1]{fontenc}
\usepackage{lmodern}
\usepackage{xspace}

\setlength{\parindent}{15pt}

\DeclareUnicodeCharacter{00A0}{~}

\begin{document}

\begin{letter}{^RECIPIENT}
  
^SENDER

% pour que la signature passe
\setlength{\sigspace}{-0.1cm}

\opening{^RANK}

Actuellement en première année de BTS IRIS (\emph{Informatique et réseau pour
l'Industrie et les Services techniques}), je souhaiterai réaliser un stage de
dévelopement informatique dans votre entreprise du ^BEGINDATE au ^ENDDATE, soit
une période de ^PERIOD.

Bien que n'étant qu'en première année de BTS, je suis déjà fort familier
avec la conception et le développement d'applications et de sites Internet.
Je suis formé à la méthode Merise, au langage SQL et possède de bonnes
compétences en C, C++, Perl, HTML, CSS, PHP, JavaScript, WLangage et utilise
quotidiennement les systèmes UNIX (Linux, Mac OS X) et Windows. \\[0.1\baselineskip]
%
\indent De plus ma formation en informatique industrielle me permet d'étudier
la gestion de processus temps réel, la méthode UML et la programmation sur
systèmes embarqués. Les notions de Programmation Orientée Objet et le langage
C++ font aussi partie du programme de BTS IRIS.

Passionné et autodidacte, je suis capable de me former seul aux nouvelles
technologies et je sais la plupart du temps trouver par moi-même une solution
aux problèmes que je peux rencontrer. \\[0.1\baselineskip]
\indent De ce fait, si votre entreprise utilise une technologie, un langage,
un framework particulier qui m'est inconnu, il m'est tout à fait possible de
m'autoformer afin d'être productif le jour du stage.

\closing{%
Je me tiens à votre disposition pour convenir d'un rendez-vous. \\
Dans l'attente d'une réponse favorable de votre part, veuillez
agréer^RANKCLOSE'expression de mes salutations distinguées.
}

\end{letter}

\end{document}
